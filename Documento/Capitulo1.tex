\documentclass[12pt,a4paper]{book}
\usepackage[latin1]{inputenc}
\usepackage{amsmath}
\usepackage{amsfonts}
\usepackage{amssymb}
\usepackage{makeidx}
\usepackage{graphicx}
\usepackage{lmodern}
\usepackage{kpfonts}
\usepackage{fourier}
\usepackage[left=2cm,right=2cm,top=2cm,bottom=2cm]{geometry}
\author{Fausto Ortega}
\begin{document}

\chapter[Capitulo 1]{Introducci\'on}

\begin{normalsize}
Este trabajo de grado ha sido realizado con el \textit{Grupo de Investigaci\'on en Sistemas Inteli-
gentes de la Universidad T\'ecnica del Norte (GISI-UTN)}.
\end{normalsize}

\section{Problema}

\begin{normalsize}
El uso de sistemas  controlados por computador tiene un incremento dr\'astico en la vida diaria. Procesador y microcontroladores son incrustados cada ves mas en los dispositivos que se usan en la cotidianidad.
Debido a las restricciones de costo, muchos de estos dispositivos que corren aplicaciones de control son dise\-nados con bajo espacio, peso y restricciones de energ\'ia.
\end{normalsize}

\hfill \break

\begin{normalsize}
El desarrollo de algoritmos de control efectivo para varios sistemas din\'amicos sigue siendo relevante para la teor\'ia de control moderna.
Durante las ultimas d\'ecadas, la programaci\'on del tiempo de procesador ha sido un \'area de investigaci\'an muy act\'iva y se han desarrollado varios m\'etodos y modelos de programaci\'on diferentes.
\end{normalsize}

\hfill \break

\begin{normalsize}
Las caracter\'isticas de tiempo real en tareas de control efectivas para varios sistemas din\'amicos sigue siendo relevante para la teor\'ia de control moderna.
Durante las ultimas d\'ecadas, la programaci\'on del tiempo de procesador ha sido un \'area de investigaci\'on muy activa y se han desarrollado varios m\'etodos y modelos de programaci\'on diferentes.
\end{normalsize}

\hfill \break\\

\begin{normalsize}
Las caracter\'isticas de tiempo real en tareas con comportamiento din\'amico, junto con restricciones de costo y recursos, crea nuevos problemas que deben abordarse en el dise\-no de dichos sistemas, en diferentes niveles de arquitectura. Algunos investigadores sugieren que se deben usar nuevas formas de control para distribuir los recursos adecuadamente en base a regulaciones de control orientadas a los recursos como el control disparado por eventos y el control auto-disparado.
\end{normalsize}

\hfill \break\\

\begin{normalsize}
Para muchos sistemas de control, el rendimiento depende en gran medida de las variaciones de retardo en las tareas de control. Dichas variaciones pueden provenir de numerosas fuentes, incluidas la prioridad de las tareas, las variaciones en las cargas de trabajo de las tareas, errores de medici\'on y las perturbaciones en el entorno f\'isico, y pueden causar un rendimiento del sistema de control degradado, como una respuesta lenta y un comportamiento err\'oneo. Ademas las propiedades de los algoritmos de programaci\'on en tiempo real pueden causar repuestas inesperadas del sistema de control en la implementaci\'on de sistemas controlados por procesador en tiempo real. Despu\'es de que una tarea ha sido lanzada, tiene que retazar su inicio de ejecuci\'on, la acci\'on de control tambi\'en puede ser remplazada o bloqueada al intentar acceder a recursos compartidos del procesador, esto significa que los instantes de tiempo de ejecuci\'on de la tarea de control no son equidistas en el tiempo  de ejecuci\'on.
\end{normalsize}


\section{Alcance}
\begin{normalsize}
En el presente proyecto se considerara un modelo en espacio de estados que represente cualquier tipo de planta, donde las matrices de la din\'amica se encuentran en funci\'on del tiempo de muestreo y el retardo.  Ante esto se propone un modelo que considere que el tiempo de muestreo no se mida entre tomas de datos de los sensores sino entre puntos de actuaci\'on manteniendo la periodicidad, donde el modelo en espacio de estados discreto depende del tiempo de muestreo, de los retardos del tiempo entre actuaci\'on, permitiendo que a pesar de que existan retardos siempre se tenga un buen desempe\-no, es decir robustez. Se evaluara el modelo de control con simulaciones a trav\'es de software matem\'atico.
\end{normalsize}

\section{Objetivos}

\subsection{Objetivo General}

\begin{normalsize}
Dise\-nar un algoritmo para disminuci\'on de degradacion del rendimiento en sistemas de control considerando actuaci\'on periodica.
\end{normalsize}

\subsection{Objetivos Espec\'ificos}

\begin{itemize} 
\item Proponer un modelo de control para disminuci\'on de degradaci\'on del rendimiento en sistemas de control, basado en la literatura. 
\item Desarrollar un algoritmo para la implementaci\'on del modelo de control. 
\item Evaluar el desempe\-no del metodo propuesto. 
\end{itemize}





\section{Justificaci\'on}
\begin{normalsize}
Los sistemas de control digital constituyen una gran parte de todos los sistemas en tiempo real.
A pesar de esto, sorprendentemente se ha hecho poco esfuerzo para estudiar su comportamiento oportuno cuando se implementa como tareas peri\'odicas en su computador. Para muchos sistemas ciber-f\'isicos, la coordinaci\'on de inteligencia entre el dise\-no de control y la implementaci\'on de su computadora correspondiente pueden llevar a su mejor rendimiento de control y/o una reducci\'on de costos. La mayor\'ia de los dispositivos integrados interact\'uan con el entorno y tienen especificaciones de calidad exigentes, cuya satisfacci\'on requiere que el sistema reaccione de manera oportuna a eventos  externos y ejecute actividades computacionales dentro de restricciones de tiempo precisas.

\hfill \break\\

Los sistemas de control basados en computadora utilizan t\'ecnicas en tiempo real para resolver problemas reales de ingenier\'ia en mecatr\'onica. Los fundamentos te\'oricos obtenidos se convierten en las bases del dise\-no rob\'otico y mecatr\'onico brindando la capacidad de an\'alisis y razonamiento matem\'atico para detectar, analizar y resolver problemas de ingenier\'ia que involucren la mec\'anica, la microelectr\'onica, la rob\'otica y biomec\'anica , desarrollando habilidades para una actualizaci\'on permanente a lo largo del ejercicio profesional.

\end{normalsize}

\end{document}

